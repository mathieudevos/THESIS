\chapter{Introduction}

With the Internet evolving at an ever increasing rate it seems to be ossifying. Is this evolution actually a positive aspect though? With a plethora of new protocols, legacy protocols, \ldots it would seem that a fundamental look needs to be taken at the current Internet. Starting at the origin of the Internet we move through the history and note that several issues have risen with the Internet that have never been properly addressed. We note that items such as: multihoming, layer boundary violation, \ldots are all issues that have been talked about, but never been resolved. The current Internet is a slow, outdated design that needs constant patching to properly function instead of truely evolving. An alternative is needed. 

\npar

John Day noticed this problem with the current Internet and decided to do some research to what could be changed for this. This lead him to the conclusion that:

\begin{quote}
	\emph{'Networking is Inter Process Communication (IPC) and IPC only' \citep{johnday2008}}
\end{quote}

Based on this a new architecture was developed. This architecture has been named: \textbf{Recursive InterNetwork Architecture}, or in short: \textbf{RINA}. This new design takes us back to a very simple basic concept and builds recursively further on this. Old issues that still plague the current Internet vanish with this new design. The essence of the idea is quite simple: use only one repeating layer, a DIF. Every IPC is provided by IPC processes. These IPC processes are brought together in a Distributed IPC Facility, a DIF. This DIF provides IPC servives over the DIF's scope to the distributed applications above. The DIF below is the point of attachment. When applications on the machines want to communicate they utilise these DIFs to set up communication, no static amount of DIFs is required.
\npar
The IRATI project has a clear goal to design a basic functional RINA. This means that low level layer support needs to be provided. A working Ethernet implementation has been realized already. In recent light it has become clear that the Internet is moving towards mobility instead of the static desktops. An implementation is needed for WiFi. This is where this Master Thesis is situated. It will try to provide RINA over WiFi. Added to this all is implementation for mobile devices, thus this thesis will handle WiFi on Android. The choice for Android was quickly made as it is partially open source and thus most of the code is publicly available.
\npar
In this thesis we will start with a literature study that will take us through the history of the Internet. Following this brief history, the study will state the research question and provide the basic information on where to start the actual research. Once the literature study has been concluded with its placement in the Master Thesis, we will describe the specification of RINA over WiFi. The next step is the actual implementation of the current IRATI prototype on an Android test device. Finally a conclusion will be written with added notes for future steps in this project.